
\documentclass[letterpaper,11pt]{article}

\usepackage{latexsym}
\usepackage[empty]{fullpage}
\usepackage{titlesec}
\usepackage{marvosym}
\usepackage[usenames,dvipsnames]{color}
\usepackage{verbatim}
\usepackage{enumitem}
\usepackage[hidelinks]{hyperref}
\usepackage{fancyhdr}
\usepackage[english]{babel}
\usepackage{tabularx}
\input{glyphtounicode}


%----------FONT OPTIONS----------
% sans-serif
% \usepackage[sfdefault]{FiraSans}
% \usepackage[sfdefault]{roboto}
% \usepackage[sfdefault]{noto-sans}
% \usepackage[default]{sourcesanspro}

% serif
% \usepackage{CormorantGaramond}
% \usepackage{charter}


\pagestyle{fancy}
\fancyhf{} % clear all header and footer fields
\fancyfoot{}
\renewcommand{\headrulewidth}{0pt}
\renewcommand{\footrulewidth}{0pt}

% Adjust margins
\addtolength{\oddsidemargin}{-0.5in}
\addtolength{\evensidemargin}{-0.5in}
\addtolength{\textwidth}{1in}
\addtolength{\topmargin}{-.5in}
\addtolength{\textheight}{1.0in}

\urlstyle{same}

\raggedbottom
\raggedright
\setlength{\tabcolsep}{0in}

% Sections formatting
\titleformat{\section}{
  \vspace{-4pt}\scshape\raggedright\large
}{}{0em}{}[\color{black}\titlerule \vspace{-5pt}]

% Ensure that generate pdf is machine readable/ATS parsable
\pdfgentounicode=1

%-------------------------
% Custom commands
\newcommand{\resumeItem}[1]{
  \item\small{
    {#1 \vspace{-2pt}}
  }
}

\newcommand{\resumeSubheading}[4]{
  \vspace{-2pt}\item
    \begin{tabular*}{0.97\textwidth}[t]{l@{\extracolsep{\fill}}r}
      \textbf{#1} & #2 \\
      \textit{\small#3} & \textit{\small #4} \\
    \end{tabular*}\vspace{-7pt}
}

\newcommand{\resumeSubSubheading}[2]{
    \item
    \begin{tabular*}{0.97\textwidth}{l@{\extracolsep{\fill}}r}
      \textit{\small#1} & \textit{\small #2} \\
    \end{tabular*}\vspace{-7pt}
}

\newcommand{\resumeProjectHeading}[2]{
    \item
    \begin{tabular*}{0.97\textwidth}{l@{\extracolsep{\fill}}r}
      \small#1 & #2 \\
    \end{tabular*}\vspace{-7pt}
}

\newcommand{\resumeSubItem}[1]{\resumeItem{#1}\vspace{-4pt}}

\renewcommand\labelitemii{$\vcenter{\hbox{\tiny$\bullet$}}$}

\newcommand{\resumeSubHeadingListStart}{\begin{itemize}[leftmargin=0.15in, label={}]}
\newcommand{\resumeSubHeadingListEnd}{\end{itemize}}
\newcommand{\resumeItemListStart}{\begin{itemize}}
\newcommand{\resumeItemListEnd}{\end{itemize}\vspace{-5pt}}

%-------------------------------------------
%%%%%%  RESUME STARTS HERE  %%%%%%%%%%%%%%%%%%%%%%%%%%%%


\begin{document}

%----------HEADING----------
% \begin{tabular*}{\textwidth}{l@{\extracolsep{\fill}}r}
%   \textbf{\href{http://sourabhbajaj.com/}{\Large Sourabh Bajaj}} & Email : \href{mailto:sourabh@sourabhbajaj.com}{sourabh@sourabhbajaj.com}\\
%   \href{http://sourabhbajaj.com/}{http://www.sourabhbajaj.com} & Mobile : +1-123-456-7890 \\
% \end{tabular*}

\begin{center}
    \textbf{\Huge \scshape Allef Rodrigo Schmidt} \\ \vspace{1pt}
     \href{mailto:x@x.com}{\underline{dev.allefschmidt@gmail.com}} $|$ 
    \href{https://linkedin.com/in/...}{\underline{https://www.linkedin.com/in/schmidt-maker/}} $|$
    \href{https://github.com/...}{\underline{https://github.com/allefrodrigo}}
\end{center}




%-----------EXPERIENCE-----------
\section{Experiência}
  \resumeSubHeadingListStart

    \resumeSubheading
      {Game Developer}{Março 2024 -- Presente}
      {Apetit Studios}{}
      \resumeItemListStart
        \resumeItem{Participação em todas as etapas de produção de jogos, do design ao lançamento.}
        \resumeItem{Desenvolvimento de jogos 2D com Godot Engine.}
        \resumeItem{Integração de efeitos sonoros e trilhas para melhorar a imersão.}
        \resumeItem{Otimização de performance para várias plataformas.}
        \resumeItem{Coordenação de testes beta e ajustes de mecânicas.}
        \resumeItem{Documentação detalhada do desenvolvimento.}
        \resumeItem{Colaboração com equipes multidisciplinares.}
      \resumeItemListEnd
      
  \resumeSubHeadingListEnd

%-----------EXPERIENCE-----------
  \resumeSubHeadingListStart

    \resumeSubheading
      {Desenvolvedor Fullstack}{Setembro 2021 -- Presente}
      {IDSOFT}{}
      \resumeItemListStart
        \resumeItem{Análise de requisitos e programação para Android e iOS usando React Native.}
        \resumeItem{Desenvolvimento de aplicações mobile focadas em desempenho e usabilidade.}
        \resumeItem{Melhorias de desempenho em sistemas web com React.js (front-end) e Node.js (back-end).}
        \resumeItem{Manutenção de APIs RESTful e integração com bancos de dados SQL e NoSQL (SQL, MongoDB).}
        \resumeItem{Uso de Git para versionamento e colaboração.}
        \resumeItem{Desenvolvimento com Vite para otimização de performance.}
        \resumeItem{Correções e ajustes para atender às necessidades dos usuários/clientes.}
      \resumeItemListEnd
      
  \resumeSubHeadingListEnd
      
  \resumeSubHeadingListStart

    \resumeSubheading
      {Desenvolvedor Frontend}{Agosto 2021 -- Junho 2022}
      {Núcleo Acesso à Terra Urbanizada}{}
      \resumeItemListStart
        \resumeItem{Construção da estética da plataforma digital REURB usando Vue.js para melhorar a experiência do usuário.}
        \resumeItem{Desenvolvimento do módulo de visualização e interação dinâmica de mapas com Leaflet.}
        \resumeItem{Criação de um dashboard para monitoramento de dados em tempo real, otimizando processos e monitorando o desempenho do projeto.}
      \resumeItemListEnd
  \resumeSubHeadingListEnd
  \resumeSubHeadingListStart

       \resumeSubheading
      {Desenvolvedor Frontend}{Jun 2021 -- Jan 2022}
      {Agrícola Famosa S/A}{}
      \resumeItemListStart
        \resumeItem{Manutenção e melhorias em aplicações existentes, focando em balanças utilizando microcomputador Raspberry Pi.}
        \resumeItem{Aprimoramento do design da aplicação para melhorar a experiência do usuário.}
        \resumeItem{Implementação de técnicas para otimizar o consumo e envio de dados via API REST.}
        \resumeItem{Trabalho com ferramentas como React.js, Node.js, e Git para desenvolvimento e versionamento.}
      \resumeItemListEnd
        \resumeSubHeadingListEnd
  \resumeSubHeadingListStart
       \resumeSubheading
      {Product Manager}{Jan 2019 -- Jul 2021}
      {Maker Educ}{}
      \resumeItemListStart
        \resumeItem{Implementação de novas tecnologias para obter vantagem competitiva no mercado.}
      
        \resumeItem{Desenvolvimento de aplicações mobile usando React Native.}
        \resumeItem{Utilização de Git para versionamento e colaboração.}
        \resumeItem{Programação com Python para funcionalidades específicas do aplicativo.}
      \resumeItemListEnd
  \resumeSubHeadingListEnd            

  

%-----------PROJECTS-----------
\section{Principais Projetos}
    \resumeSubHeadingListStart
      \resumeProjectHeading
          {\textbf{MakerApp,}{ Maker Educ} $|$ \emph{React Native, Git, Python, Typescript}}{}
          \resumeItemListStart
            \resumeItem{Criação do aplicativo Maker Educ, um aplicativo multiplataforma que permite a programação do microcontrolador Atmega328P diretamente do celular do estudante, eliminando a necessidade de um computador para aulas de programação/robótica.}
          \resumeItemListEnd
          
      \resumeProjectHeading
           {\textbf{Plataforma Monitora Caju,}{ Embrapa - CE} $|$ \emph{React Native, Typescript, Next.js, MongoDB}}{}
          \resumeItemListStart
            \resumeItem{Aplicativo desenvolvido para ajudar produtores de caju a cuidar de suas plantações, prevenindo doenças e aumentando a produção.}
            \resumeItem{O sistema inclui métodos CRUD, autenticação, mapas de calor mostrando a incidência de doenças no mapa e análise de dados com dashboard.}
          \resumeItemListEnd
          
              \resumeProjectHeading
          {\textbf{Plataforma de Gestão de Construção,}{ Pinte Pinturas } $|$ \emph{Typescript}}{}
          \resumeItemListStart
            \resumeItem{Desenvolvimento de uma plataforma web que melhorou significativamente a eficiência organizacional de uma grande empresa de construção, substituindo um sistema baseado em Excel.}
            \resumeItem{Sistema permitiu acompanhamento em tempo real, geração eficiente de relatórios e acesso a dashboards dinâmicos para insights instantâneos.}
            \resumeItem{Transformação que otimizou processos, gerenciou recursos de forma mais eficaz, assegurou conclusão oportuna dos projetos e facilitou a tomada de decisões informadas.}
          \resumeItemListEnd
          
         \resumeProjectHeading
     {\textbf{Plataforma Digital REURB,} {Governo Federal} $|$ \emph{Vue.js, Quasar UI, Axios}}{}
          \resumeItemListStart
         \resumeItem{Desenvolvimento do módulo de visualização e interação dinâmica de mapas utilizando Leaflet.}
            \resumeItem{Criação de um dashboard em Vue.js para monitoramento de dados em tempo real, otimizando processos e monitorando o desempenho e crescimento do projeto.}
            \resumeItem{Sistema com métodos CRUD, autenticação, mapas + GeoJSON, gráficos e dashboards.}
          \resumeItemListEnd
    \resumeSubHeadingListEnd


%-----------EDUCATION-----------
\section{Educação}
  \resumeSubHeadingListStart
    \resumeSubheading
      {Universidade Federal Rural do Semi-Árido}{Mossoró, RN}
      {Bacharelado em Ciência da Computação}{2015 -- 2022}
    \resumeSubheading
      {Instituto Metrópole Digital - IMD/UFRN}{Natal, RN}
      {Desenvolvedor Web}{2020 -- 2021}
      \resumeSubheading
      {EF SET}{}
      {Teste de Proficência - Inglês - C1}{}
  \resumeSubHeadingListEnd
  
%-----------TECHNICAL SKILLS-----------
\section{Habilidades técnicas}
 \begin{itemize}[leftmargin=0.15in, label={}]
    \small{\item{
     \textbf{Linguagens}{: JavaScript, Typescript, Angular, Python} \\
     \textbf{Frameworks}{: React, React Native, Vue.js, Next.js, Node.js} \\
     \textbf{Developer Tools}{: Git, Docker, Vite, TravisCI, Postman, Figma} \\
     \textbf{Bibliotecas}{: Axios, Leaflet, pandas, NumPy, Matplotlib, Redux, Lodash, Moment.js, Express.js, Chart.js, D3.js} \\
     \textbf{Banco de dados}{: PostgreSQL, MongoDB}
    }}
 \end{itemize}


%-------------------------------------------
\end{document}